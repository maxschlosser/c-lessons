%% Nothing to modify here.
%% make sure to include this before anything else

\documentclass[10pt]{beamer}
\usetheme{Szeged}
\usenavigationsymbolstemplate{}
\usepackage{tgcursor}

% packages
\usepackage{color}
\usepackage{listings}

% color definitions
\definecolor{mygreen}{rgb}{0,0.6,0}
\definecolor{mygray}{rgb}{0.5,0.5,0.5}
\definecolor{mymauve}{rgb}{0.58,0,0.82}

% re-format the title frame page
\makeatletter
\def\supertitle#1{\gdef\@supertitle{#1}}%
\setbeamertemplate{title page}
{
  \vbox{}
  \vfill
  \begin{centering}
  \begin{beamercolorbox}[sep=8pt,center]{title}
      \usebeamerfont{supertitle}\@supertitle
   \end{beamercolorbox}
    \begin{beamercolorbox}[sep=8pt,center]{title}
      \usebeamerfont{title}\inserttitle\par%
      \ifx\insertsubtitle\@empty%
      \else%
        \vskip0.25em%
        {\usebeamerfont{subtitle}\usebeamercolor[fg]{subtitle}\insertsubtitle\par}%
      \fi%     
    \end{beamercolorbox}%
    \vskip1em\par
    \begin{beamercolorbox}[sep=8pt,center]{author}
      \usebeamerfont{author}\insertauthor
    \end{beamercolorbox}
    \begin{beamercolorbox}[sep=8pt,center]{institute}
      \usebeamerfont{institute}\insertinstitute
    \end{beamercolorbox}
    \begin{beamercolorbox}[sep=8pt,center]{date}
      \usebeamerfont{date}\insertdate
    \end{beamercolorbox}\vskip0.5em
    {\usebeamercolor[fg]{titlegraphic}\inserttitlegraphic\par}
  \end{centering}
  \vfill
}
\makeatother

% insert frame number
\expandafter\def\expandafter\insertshorttitle\expandafter{%
      \insertshorttitle\hfill%
\insertframenumber\,/\,\inserttotalframenumber}

% preset-listing options
\lstset{
  backgroundcolor=\color{white},   
  % choose the background color; 
  % you must add \usepackage{color} or \usepackage{xcolor}
  basicstyle=\footnotesize,
  % the size of the fonts that are used for the code
  breakatwhitespace=false,         
  % sets if automatic breaks should only happen at whitespace
  breaklines=true,                 % sets automatic line breaking
  captionpos=b,                    % sets the caption-position to bottom
  commentstyle=\color{mygreen},    % comment style
  % deletekeywords={...},            
  % if you want to delete keywords from the given language
  extendedchars=true,              
  % lets you use non-ASCII characters; 
  % for 8-bits encodings only, does not work with UTF-8
  frame=single,                    % adds a frame around the code
  keepspaces=true,                 
  % keeps spaces in text, 
  % useful for keeping indentation of code 
  % (possibly needs columns=flexible)
  keywordstyle=\color{blue},       % keyword style
  % morekeywords={*,...},            
  % if you want to add more keywords to the set
  numbers=left,                    
  % where to put the line-numbers; possible values are (none, left, right)
  numbersep=5pt,                   
  % how far the line-numbers are from the code
  numberstyle=\tiny\color{mygray}, 
  % the style that is used for the line-numbers
  rulecolor=\color{black},         
  % if not set, the frame-color may be changed on line-breaks 
  % within not-black text (e.g. comments (green here))
  stepnumber=1,                    
  % the step between two line-numbers. 
  % If it's 1, each line will be numbered
  stringstyle=\color{mymauve},     % string literal style
  tabsize=4,                       % sets default tabsize to 4 spaces
  title=\lstname                   
  % show the filename of files included with \lstinputlisting; 
  % also try caption instead of title
}

% macro for code inclusion
\newcommand{\includecode}[2][c]{
	\lstinputlisting[caption=#2, style=custom#1]{#2}
}
	% nothing to do here
\usepackage[english]{babel}

\usepackage[utf8]{inputenc}

\newcommand{\course}{
	C advanced
}

\author{
	Richard Mörbitz,
	Manuel Thieme
}

\lstset{
	language = C,
	showspaces = false,
	showtabs = false,
	showstringspaces = false,
	escapechar = @,
	belowskip=-1.5em
}

\def\ContinueLineNumber{\lstset{firstnumber=last}}
\def\StartLineAt#1{\lstset{firstnumber=#1}}
\let\numberLineAt\StartLineAt

\makeatletter
\def\beamer@verbatimreadframe{%
	\begingroup%
	\let\do\beamer@makeinnocent\dospecials%
	\count@=127%
	\@whilenum\count@<255 \do{%
		\advance\count@ by 1%
		\catcode\count@=11%
	}%
	\beamer@makeinnocent\^^L% and whatever other special cases
	\beamer@makeinnocent\^^I% <-- PATCH: allows tabs to be written to temp file
	\endlinechar`\^^M \catcode`\^^M=12%
	\@ifnextchar\bgroup{\afterassignment\beamer@specialprocessframefirstline\let\beamer@temp=}{\beamer@processframefirstline}}%
\makeatother % TODO modify this if you have not already done so

% meta-information
\newcommand{\topic}{
	Type qualifiers
}

% nothing to do here
\title{\topic}
\supertitle{\course}
\date{}

% the actual document
\begin{document}

\maketitle

\begin{frame}{Contents}
	\tableofcontents
\end{frame}

\section{Why preprocessing?}
\subsection{}

\begin{frame}{Limits of a language}
	The C language is mainly used for low-level applications working directly\\
	with the operating system or even hardware.\\
	\bigskip
	Due to that, C programs must be able to:
	\begin{itemize}
		\item use different system calls based on the operating system
		\item link against different libraries based on the target system
		\item compile to different machine instructions based on the instruction set\\
			  of the target machine
		\item be compiled by different compilers, etc.
	\end{itemize}
	\bigskip
	As these requirements are not expressible in C, a preprocessor is needed.\\
	Note that it is not a part of the C language and has developed independently of it.
\end{frame}

\begin{frame}[fragile]{How it's done}
	Every C compiler has a preprocesser facility that processes a source file\\
	in four phases:\\
	\begin{enumerate}
		\item Replace all trigraph sequences with the actual characters
		\begin{itemize}
			\item e.g., \textit{??(} becomes \textit{[}
		\end{itemize}
		\item Reconnect lines that were spliced by a \textit{\textbackslash} at the end
		\item Tokenize the input and replace comments with whitespace
		\item Handle preprocessor directives and expand macros
	\end{enumerate}
	\bigskip
	To view the result of preprocessing, pass the \textit{-E} switch to your compiler:
	\begin{lstlisting}[numbers=none]
$ gcc -E <source_file>
\end{lstlisting}
\end{frame}

\section{Preprocessor directives}
\subsection{}

\begin{frame}[fragile]{Inclusion of other files}
	The well-known \textit{\#include} directive tells the preprocessor to insert
	another file in the current source file.\\
	\bigskip
	You can use angle brackets or quotes to determine where to look for a file:
	\begin{lstlisting}[numbers=none]
#include <stdio.h>	/* search the include path */
#include "util.h"	/* search the local directory first */
	   /* (falls back to the global include on failure) */
\end{lstlisting}
	It is better practice to always use the angle brackets and pass your local
	headers as additional include paths to the compiler:
	\begin{lstlisting}[numbers=none]
$ gcc src/main.c -I./include
\end{lstlisting}
	\bigskip
	You can include files of all types and names, but the \textit{.h} extension
	is the usual convention in C.
\end{frame}

\begin{frame}[fragile]{Macros without parameters}
	The \textit{\#define} directive is used to replace a token with any text.
	\begin{lstlisting}[numbers=none]
#define M_PI 3.14159265358979323846	/* found in math.h */
\end{lstlisting}
	\bigskip
	Macros are useful as a layer of abstraction, e.g. to define constants in system headers.
	The standard also introduces some predefined macros:
	\begin{itemize}
		\item \_\_FILE\_\_ and \_\_LINE\_\_: file name and current line number
		\item \_\_DATE\_\_ and \_\_TIME\_\_: current date and time
		\item \_\_STDC\_\_: 1, if the compiler conforms to standard C
		\item \_\_STDC\_VERSION\_\_: version of the C standard in use
	\end{itemize}
	\bigskip
	Macros allow all sorts of nasty obfuscation, like \textit{for ever} loops:
	\begin{lstlisting}[numbers=none]
#define ever (;;)
\end{lstlisting}
\end{frame}

\begin{frame}[fragile]{Macros with parameters}
	Macros can take any number of arguments, as if they were functions.
	\begin{lstlisting}[numbers=none]
#define DEGREES(angle_rad) (angle_rad * 180 / M_PI)
\end{lstlisting}
	\bigskip
	C99 introduced variadic macros for an unknown number of arguments.
	\begin{lstlisting}[numbers=none]
#define DEBUG(format, ...) \
	fprintf(stderr, "[%s:%s] " format, \
			__FILE__, __LINE__, __VA_ARGS__)
\end{lstlisting}
	\bigskip
	Note that you must use a \textit{\textbackslash} before each line split in a directive.\\
	\bigskip
	The C standard forbids to have preprocessor directives in the parameter list
	of macros, but the GNU C preprocessor (\textit{cpp}) allows it.
\end{frame}

\begin{frame}[fragile]{Operators in directives}
	Token stringification (\textit{\#}) converts tokens to C strings.
	\begin{lstlisting}[numbers=none]
#define STR(str) #str
const char *foo_str = str(foo);	/* becomes "foo" */
\end{lstlisting}
	\bigskip
	Token concatenation (\textit{\#\#}) combines two tokens.
\begin{lstlisting}[numbers=none]
#define CONCAT(a, b) a##b
int num = CONCAT(12, 34);	/* becomes 1234 */
\end{lstlisting}
	\bigskip
	These mechanisms can be combined to create powerful abbreviations.\\
	\bigskip
	You can let the preprocessor generate large amounts of code for you,
	an example use case are X-Macros.
\end{frame}

\begin{frame}[fragile]{Conclusion on macros}
	To remove a macro, simply \textit{\#undef} it. Macros also can be re-\textit{\#defined}.\\
	\bigskip
	Macros have a huge amount of pitfalls:
	\begin{itemize}
		\item Precedence issues when not using parentheses around arguments
		\item Possible side effects when they reference variables, use semicolons
			  or you pass increment/decrement expressions
		\item Completely surprising behaviour when they are nested
		\item Error messages become from macro expansions quite cryptic
	\end{itemize}
	\bigskip
	$\rightarrow$ use macros as rarely as possible and name them in capital letters.\\
	\bigskip
	However, some libraries intentionally make macros look like functions to have a transparent
	interface (e.g., the Lua API).
\end{frame}

\begin{frame}[fragile]{Conditional compilation}
	The preprocessor knows the conditional \textit{\#if}, \textit{\#elif}, \textit{\#else}
	and \textit{\#endif} with the usual semantics. They take expressions which can contain:
	\begin{itemize}
		\item Integer and character constants
		\item The usual arithmetic and logical operators from C
		\item Macros (expanded before evaluation)
		\item The \textit{\#defined} operator
		\item Identifiers (evaluated as 0 if they are not a macro)
	\end{itemize}
	\bigskip
	If an \textit{\#if} expression evaluates to 0, all code controlled by it is skipped.
	\begin{lstlisting}[numbers=none]
#if 0
const char *this_code = "doesn't matter";
#endif
\end{lstlisting}
\end{frame}

\begin{frame}[fragile]{Conditional compilation ctd.}
	The special operator \textit{\#defined} takes a macro and evaluates to 1, if the macro
	is currently defined. It has the shortcuts \textit{\#ifdef} and \textit{\#ifndef}.\\
	\bigskip
	These statements are commonly used for include guards and target-dependent compilation.
	\begin{lstlisting}[numbers=none]
#ifndef MY_HEADER_H
# define MY_HEADER_H
# ifdef WIN32
#  include <windows.h>
   /* expand macros to Windows API calls */
# else
#  include <unistd.h>
   /* expand macros to Unix system calls */
# endif
#endif
\end{lstlisting}
\end{frame}

\begin{frame}{Other directives}
	The standard describes additional directives, such as
	\begin{itemize}
		\item \#error $<$msg$>$: end compilation with output $<$msg$>$
		\item \#line $<$nr$>$ $[<$file$>]$: manipulate the \_\_LINE\_\_ /\_\_FILE\_\_ macros
		\item \#pragma: provide additional information to the compiler
		\begin{itemize}
			\item once: alternative to include guards, not part of the standard
			\item omp: instruct the compiler to  use OpenMP features
			\item the standard defines some pragmas, too
		\end{itemize}
	\end{itemize}
	Compilers may introduce additional directives such as GNU's \textit{\#warning}.\\
	\bigskip
	When using pragmas excessively, you gain a lot of additional expressive power, but you
	bind yourself to a certain compiler. Portable programs avoid them completely.
\end{frame}

% nothing to do from here on
\end{document}
