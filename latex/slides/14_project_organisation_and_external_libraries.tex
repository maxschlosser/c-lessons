%% Nothing to modify here.
%% make sure to include this before anything else

\documentclass[10pt]{beamer}
\usetheme{Szeged}
\usenavigationsymbolstemplate{}
\usepackage{tgcursor}

% packages
\usepackage{color}
\usepackage{listings}

% color definitions
\definecolor{mygreen}{rgb}{0,0.6,0}
\definecolor{mygray}{rgb}{0.5,0.5,0.5}
\definecolor{mymauve}{rgb}{0.58,0,0.82}

% re-format the title frame page
\makeatletter
\def\supertitle#1{\gdef\@supertitle{#1}}%
\setbeamertemplate{title page}
{
  \vbox{}
  \vfill
  \begin{centering}
  \begin{beamercolorbox}[sep=8pt,center]{title}
      \usebeamerfont{supertitle}\@supertitle
   \end{beamercolorbox}
    \begin{beamercolorbox}[sep=8pt,center]{title}
      \usebeamerfont{title}\inserttitle\par%
      \ifx\insertsubtitle\@empty%
      \else%
        \vskip0.25em%
        {\usebeamerfont{subtitle}\usebeamercolor[fg]{subtitle}\insertsubtitle\par}%
      \fi%     
    \end{beamercolorbox}%
    \vskip1em\par
    \begin{beamercolorbox}[sep=8pt,center]{author}
      \usebeamerfont{author}\insertauthor
    \end{beamercolorbox}
    \begin{beamercolorbox}[sep=8pt,center]{institute}
      \usebeamerfont{institute}\insertinstitute
    \end{beamercolorbox}
    \begin{beamercolorbox}[sep=8pt,center]{date}
      \usebeamerfont{date}\insertdate
    \end{beamercolorbox}\vskip0.5em
    {\usebeamercolor[fg]{titlegraphic}\inserttitlegraphic\par}
  \end{centering}
  \vfill
}
\makeatother

% insert frame number
\expandafter\def\expandafter\insertshorttitle\expandafter{%
      \insertshorttitle\hfill%
\insertframenumber\,/\,\inserttotalframenumber}

% preset-listing options
\lstset{
  backgroundcolor=\color{white},   
  % choose the background color; 
  % you must add \usepackage{color} or \usepackage{xcolor}
  basicstyle=\footnotesize,
  % the size of the fonts that are used for the code
  breakatwhitespace=false,         
  % sets if automatic breaks should only happen at whitespace
  breaklines=true,                 % sets automatic line breaking
  captionpos=b,                    % sets the caption-position to bottom
  commentstyle=\color{mygreen},    % comment style
  % deletekeywords={...},            
  % if you want to delete keywords from the given language
  extendedchars=true,              
  % lets you use non-ASCII characters; 
  % for 8-bits encodings only, does not work with UTF-8
  frame=single,                    % adds a frame around the code
  keepspaces=true,                 
  % keeps spaces in text, 
  % useful for keeping indentation of code 
  % (possibly needs columns=flexible)
  keywordstyle=\color{blue},       % keyword style
  % morekeywords={*,...},            
  % if you want to add more keywords to the set
  numbers=left,                    
  % where to put the line-numbers; possible values are (none, left, right)
  numbersep=5pt,                   
  % how far the line-numbers are from the code
  numberstyle=\tiny\color{mygray}, 
  % the style that is used for the line-numbers
  rulecolor=\color{black},         
  % if not set, the frame-color may be changed on line-breaks 
  % within not-black text (e.g. comments (green here))
  stepnumber=1,                    
  % the step between two line-numbers. 
  % If it's 1, each line will be numbered
  stringstyle=\color{mymauve},     % string literal style
  tabsize=4,                       % sets default tabsize to 4 spaces
  title=\lstname                   
  % show the filename of files included with \lstinputlisting; 
  % also try caption instead of title
}

% macro for code inclusion
\newcommand{\includecode}[2][c]{
	\lstinputlisting[caption=#2, style=custom#1]{#2}
}
	% nothing to do here
\usepackage[english]{babel}

\usepackage[utf8]{inputenc}

\newcommand{\course}{
	C advanced
}

\author{
	Richard Mörbitz,
	Manuel Thieme
}

\lstset{
	language = C,
	showspaces = false,
	showtabs = false,
	showstringspaces = false,
	escapechar = @,
	belowskip=-1.5em
}

\def\ContinueLineNumber{\lstset{firstnumber=last}}
\def\StartLineAt#1{\lstset{firstnumber=#1}}
\let\numberLineAt\StartLineAt

\makeatletter
\def\beamer@verbatimreadframe{%
	\begingroup%
	\let\do\beamer@makeinnocent\dospecials%
	\count@=127%
	\@whilenum\count@<255 \do{%
		\advance\count@ by 1%
		\catcode\count@=11%
	}%
	\beamer@makeinnocent\^^L% and whatever other special cases
	\beamer@makeinnocent\^^I% <-- PATCH: allows tabs to be written to temp file
	\endlinechar`\^^M \catcode`\^^M=12%
	\@ifnextchar\bgroup{\afterassignment\beamer@specialprocessframefirstline\let\beamer@temp=}{\beamer@processframefirstline}}%
\makeatother % TODO modify this if you have not already done so

% meta-information
\newcommand{\topic}{
	Project organisation and external libraries
}

% nothing to do here
\title{\topic}
\supertitle{\course}
\date{}

% the actual document
\begin{document}

\maketitle

\begin{frame}{Contents}
	\tableofcontents
\end{frame}

\section{Header files}
\subsection{}
\begin{frame}{It is a mess}
	The current version of our Dungeon is \textit{2\_finished/main.c}. In this file, there are:
	\begin{itemize}
		\item 205 lines of code
		\item 4 includes
		\item 2 custom enums
		\item 3 custom data type definitions
		\item 4 global variables
		\item 14 functions and its prototypes
		\item And the main function
	\end{itemize}\ \\\ \\
	We now will organize the project in multiple files.
\end{frame}

\begin{frame}[fragile]{Header files}
	The first thing we do is to seperate the I/O part into a file called \textit{io.c}:\\
	\begin{itemize}
		\item \textit{handle\_input, draw\_board, print\_entity}
	\end{itemize}
	Since we are calling the I/O functions in our \textit{main.c} file, the prototypes have to stay accessible from there.\\
	This is why we are creating our own \textbf{header file} \textit{io.h}, containing the prototypes, and we include it in our \textit{main.c} and \textit{io.c}:\\\ \\
	\begin{lstlisting}
#include "io.h"
\end{lstlisting}\ \\\ \\
	\begin{itemize}
		\item Note that we use \textbf{"\ldots"} for own includes instead of $\boldsymbol{<\ldots>}$.
		\item The quotation marks contain the relative path to the header file.
	\end{itemize}
\end{frame}

\begin{frame}{Data structures}
	The next step is to seperate all functions concerning the custom data structures to \textit{datastructures.c}:
	\begin{itemize}
		\item init\_entity
		\item init\_monster\_list
		\item free\_monster\_list
		\item add\_monster
		\item monster\_list\_map
		\item position\_covered
		\item has\_position
	\end{itemize}\ \\\ \\
	We put the type definition of our structs with the prototypes into the header file \textit{datastructures.h}.

\end{frame}

\begin{frame}[fragile]{Global variables}
	Since the source files all are accessing the global variables, we move them into another header file \textit{globals.h}.\\\ \\
	Finally we outsource the function \textit{out\_of\_bounds} and the global variable \textit{next\_coords} into \textit{board.c}.\\
	\begin{itemize}
		\item Try to find the remaining includes by yourself.
	\end{itemize}\ \\\ \\
	To avoid multiple includes of one header file we can, and should, do as follows:
	\begin{lstlisting}
#ifndef BOARD_H
#define BOARD_H

/* header file content here */

#endif /* BOARD_H */
\end{lstlisting}
\end{frame}

\begin{frame}[fragile]{Where to put these files?}
Now we have a mess of \textbf{.h} and \textbf{.c} files hanging around in a single directory.\\
Best practice is to:
\begin{itemize}
	\item Put all header files into an \textit{inc} or \textit{include} directory
	\item Put all source files into a \textit{src} or \textit{source} directory
\end{itemize}
\pause
\bigskip
But wait! Now we have to write awful relative paths into every file:
\begin{lstlisting}
#include "../include/foo.h"
\end{lstlisting}
\pause
\bigskip
You can pass additional include paths to gcc with the \textbf{-I} flag:
\begin{lstlisting}
#include <foo.h>
\end{lstlisting}
\begin{lstlisting}[language=bash]
$ gcc src/foo.c -I./include
\end{lstlisting}

\end{frame}

\section{Make}
\subsection{}
\begin{frame}[fragile]{Compiling}
	To compile a project from multiple files you have to pass all source files to gcc:\\\ \\
	\begin{lstlisting}[numbers=none]
$ gcc -Wall -o main datastructures.c io.c board.c main.c
\end{lstlisting}\ \\\ \\
	\begin{itemize}
		\item Note that you do not have to add the header files.
		\item They do not need to be compiled separately.
	\end{itemize}
\end{frame}

\begin{frame}[fragile]{Makefiles}
	There is a package called \textit{make} that allows you to write the compile order into a file and start compiling with a more simple command:\\\ \\
	\begin{lstlisting}[numbers=none]
$ make
\end{lstlisting}\ \\\ \\
	To use it, you have to install the package and create a file named \textit{Makefile} in your project directory.\\
	\begin{itemize}
		\item Note that the \textbf{make} package is much more mighty than that.
	\end{itemize}
\end{frame}

\begin{frame}[fragile]{Make rulez!}
	You can define rules in your makefile:
	\begin{lstlisting}[language=make,basicstyle=\scriptsize,numbers=none]
all:
	gcc -o main datastructures.c io.c board.c main.c
debug:
	gcc -Wall -g -o main datastructures.c io.c board.c main.c
\end{lstlisting}
	Now you can build your project with:
	\begin{lstlisting}[basicstyle=\scriptsize,numbers=none]
$ make all
\end{lstlisting}
	or with
	\begin{lstlisting}[basicstyle=\scriptsize,numbers=none]
$ make debug
\end{lstlisting}
	if you want to have errors and debug information.
	\begin{itemize}
		\item Note: if you are calling make without a target, the first one is built.
	\end{itemize}
\end{frame}

\begin{frame}{About compiling}
	With many source files, we do not want to recompile the whole project if we changed a tiny piece of code.\\
	Make offers the opportunity to compile only if dependencies have changed. But first a little theory about compiling:\\\ \\
	\scriptsize
	\centering
	\begin{tabular}{|l|l|l|l|}
	\hline
	\textbf{process}	&	\textbf{input}	&	\textbf{output}										&	\textbf{notes}				\\\hline
	preprocessor		&	\textit{*.c}	&	source code with substitutions \textit{*.i}			&								\\\hline
	compiling			&	\textit{*.i}	&	assembler mnemonics \textit{*.s}					&	debug information is added	\\\hline
	assembling			&	\textit{*.s}	&	byte code \textit{*.o}								&	not executable yet			\\\hline
	linking				&	\textit{*.o}	&	executable byte code \textit{*} / \textit{*.exe}	&								\\\hline				
	\end{tabular}\ \\\ \\
	\flushleft
	\normalsize
	We would like to split the compilation process and keep the unlinked *.o files to link newly build *.o files with existing ones.
\end{frame}

\begin{frame}[fragile]{Dependencies}
	gcc compiles without linking when the \textit{-c} flag is set.\\
	You can add a dependency to your makefile by writing it behind the rule identifier. Dependencies on other rules are allowed:\\\ \\
	\begin{lstlisting}[language=make,basicstyle=\scriptsize,numbers=none]
all: firstfile.o secondfile.o 
	gcc -o main firstfile.o secondfile.o

firstfile.o: firstfile.c secondfile.h
	gcc -c firstfile.c
	
secondfile.o: secondfile.c secondfile.h
	gcc -c secondfile.c
\end{lstlisting}
	\begin{itemize}
		\item Note that both source files have the same header file as a dependency here.
	\end{itemize}
\end{frame}

\begin{frame}[fragile]{A few tutorials later}
	\begin{columns}
	\column{.5\textwidth}
	\begin{lstlisting}[language=make,basicstyle=\tiny,escapeinside=§]
CC		=gcc
SRCDIR 	=src
OBJDIR	=build
INCDIR	=inc
EXE		=spaceinvaders
SRC 	=$(wildcard src/*.c)
DEP		=$(SRC:%.c=%.d)
OBJ		=$(SRC:$(SRCDIR)/%.c=$(OBJDIR)/%.o)
INC		=-I./$(INCDIR)/
CFLAGS	=-std=c99 ${INC} -Wall -Wextra -Wpedantic -Werror
LDFLAGS	=-lncurses
VPATH 	=$(SRCDIR)

.PHONY: all clean debug

all: $(OBJ) $(EXE)

debug: CFLAGS += -g
debug: $(EXE)
\end{lstlisting}
	\column{.5\textwidth}
	\begin{lstlisting}[language=make,basicstyle=\tiny,escapeinside=§,firstnumber=20]
$(OBJ): | $(OBJDIR)
$(EXE): $(OBJ)
	$(CC) -o $@ $^ $(LDFLAGS)

-include $(DEP)

%.d: %.c
	$(CC) -MM ${INC} $*.c > $*.d
	sed -i -e 's|.*:|$(patsubst $(SRCDIR)/%,$(OBJDIR)/%,$*).o:|' $*.d

build/%.o: %.c
	$(CC) -c $(CFLAGS) -o $@ $<

clean:
	rm -f $(EXE) $(OBJ) $(DEP)
	rm -rf $(OBJDIR)

$(OBJDIR):
	mkdir $@
\end{lstlisting}
	\end{columns}
\end{frame}

\begin{frame}{Other build tools}
	make is a very low-level tool, not restricted to C/C++,
	\begin{itemize}
	  \item which is great, because it can be used for anything where certain commands have to be used to generate files,
	  \item which is bad, because make does not know anything about C/C++ and offers only a rudimentary framework to execute commands.
	\end{itemize}

	Other options
	\begin{itemize}
	  \item Autotools
	  \item CMake
	  \item Scons
	  \item …
	\end{itemize}
\end{frame}

\begin{frame}[fragile]{CMake}
	CMake stands for cross-platform make and offers a more high-level, declarative approach to building.

	CMake does not build projects itself, instead it generates the necessary project files for other tools such as Make, Visual Studio or XCode.

	\begin{lstlisting}[escapeinside=§]
cmake_minimum_required (VERSION 2.8.0)
project (MyProject)
set(CMAKE_BUILD_TYPE "Release")
list(APPEND CMAKE_C_FLAGS "-std=c11 -Wall -Werror")
target_include_directories (${CMAKE_CURRENT_SOURCE_DIR})
add_library (mylib lib.c)
add_executable (myprogram main.c foo.c)
target_link_libraries (myprogram mylib)
\end{lstlisting}
\end{frame}

\section{ncurses}
\subsection{}

\begin{frame}{An external library}
	The next step to build an awesome ASCII Dungeon is to get rid of the "enter mashing" after entering the direction.\\\ \\
	There is no way to do this cross platform with the standard library. That is why we will use an external one: \textbf{ncurses}.\\\ \\
	We prepared this for you in \textit{3\_finished/io.c}. If you want, you can try to understand the content of that file.
	\begin{itemize}
		\item Have a closer look at \url{https://de.wikibooks.org/wiki/Ncurses}
	\end{itemize}

\end{frame}

% nothing to do from here on
\end{document}
